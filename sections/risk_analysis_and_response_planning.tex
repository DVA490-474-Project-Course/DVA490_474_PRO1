%===============================================================================
\section{Risk analysis and response planning}
\label{section:risk_analysis_and_response_planning}

A thorough risk analysis was done and a response plan was created to ensure that any type of damage to the project is planned for and can be dealt with appropriately.
A table was created with the following attributes:
\begin{itemize}
    \item Potential risk: short description of a potential risk. (Only those which were deemed to have a notable impact will be covered).
    \item Probability: score from $1\text{-}5$, with $1$ being very unlikely, and $5$ being very likely.
    \item Impact: score from $1\text{-}5$, with $1$ having a very small impact, and $5$ having a very large impact.
    \item Risk score: calculated according to: $\text{Probability} \times \text{Impact}$. Score from $1\text{-}25$, with $1$ being a low priority risk, and $25$ being a critical priority risk.
    \item Strategy: which strategy is to be used to deal with the risk. There are four possible: accept, transfer, ignore, mitigate.
    \item Mitigation action: what the mitigation action entails.
\end{itemize}
The table can be seen in Table.\:\ref{tab:risk_response}.

\begin{table}[H]
    \centering
    \begin{tabularx}{\textwidth}{|X|c|c|c|X|X|} \hline
         \textbf{Potential risk}                            & \textbf{Probability}  & \textbf{Impact}   & \textbf{Risk score}   & \textbf{Strategy}     & \textbf{Mitigation action} \\ \hline
         Simulation differing substantially from reality    & 3                     & 3                 & 9                     & Accept                & N/A \\ \hline
         Reversed polarity when plugging in the battery & 3 & 5 & 15 & Mitigate & Reverse polarity protection circuit \\ \hline
         Misunderstanding between hardware and software input/output values & 2 & 5 & 10 & Mitigate & Communicate between the hardware and software teams \\ \hline
         Ordered components not arriving on time & 3 & 4 & 12 & Accept & N/A \\ \hline
         Components breaking/burning up & 4 & 5 & 20 & Mitigate & Order redundant components where it makes sense and over-dimension so that the components maximum voltage level far exceeds the operational voltage.\\ \hline
         Robot excessively shaking & 2 & 5 & 10 & Mitigate & Increasing the amount of mini wheels on the wheels \\ \hline
         Dribbler bar breaking & 2 & 3 & 6 & Mitigate & Make out of a solid material \\ \hline
         Dribbler unable to induce enough rotation on the ball & 2 & 3 & 6 & Mitigate & Powerful motor and high grip material \\ \hline
         Ball bouncing away instead of being caught in the rotation & 4 & 3 & 16 & Mitigate & Design dribbler assembly in a way that reduces bouncing \\ \hline
         Electrocution injury & 2 & 4 & 8 & Mitigate & Take proper precautions, use protective gear, and rethink the action before performing \\ \hline
         Hardware components are not equal to those used by our collaborators & 5 & 2 & 10 & Mitigate & Have a continuous dialog with our collaborators as to minimise misunderstandings \\ \hline
         Project duration is not enough to complete the original plan & 3 & 3 & 9 & Mitigate & Plan reasonably, continuously following up on progress made, not an too ambitious initial goal \\ \hline
         Team members are unwell(sick) & 4 & 3  & 12 & Accept & N/A \\ \hline
         Miscommunication in the collaboration & 4 & 3 & 10 & Mitigate  & Use clear communication channels, define roles and responsibilities, document key decisions and set regular check-ins. \\ \hline
         Simulation differing substantially from reality & 3 & 3 & 9 & Accept & N/A \\ \hline
         Reversed polarity when plugging in the battery & 3 & 5 & 15 & Mitigate & Reverse polarity protection circuit \\ \hline
         Misunderstanding between hardware and software input/output values & 2 & 5 & 10 & Mitigate & Communicate between the hardware and software teams \\ \hline
         Ordered components not arriving on time & 3 & 4 & 12 & Accept & N/A \\ \hline
         Components breaking/burning up & 4 & 5 & 20 & Mitigate & Order redundant components where it makes sense and over-dimension so that the components maximum voltage level far exceeds the operational voltage. \\ \hline
         Robot excessively shaking & 2 & 5 & 10 & Mitigate & Increase the number of mini wheels on the wheels \\ \hline
         Dribbler bar breaking & 2 & 3 & 6 & Mitigate & Make out of a solid material \\ \hline
         Dribbler unable to induce enough rotation on the ball & 2 & 3 & 6 & Mitigate & Use a powerful motor and high grip material \\ \hline
         Ball bouncing away instead of being caught in the rotation & 4 & 3 & 16 & Mitigate & Design dribbler assembly in a way that reduces bouncing \\ \hline
         Electrocution injury & 2 & 4 & 8 & Mitigate & Take proper precautions, use protective gear, and rethink the action before performing \\ \hline
         Hardware components are not equal to those used by our collaborators & 5 & 2 & 10 & Mitigate & Have a continuous dialog with our collaborators to minimise misunderstandings \\ \hline
         Project duration is not enough to complete the original plan & 3 & 3 & 9 & Mitigate & Plan reasonably, continuously follow up on progress made, and set realistic goals \\ \hline
         Team members are unwell (sick) & 4 & 3 & 10 & Accept & N/A \\ \hline
         Misunderstanding between team members & 4 & 3 & 10 & Accept & N/A \\ \hline
         Power supply or battery failure & 3 & 4 & 12 & Mitigate & Implement redundant power supplies and use a reliable battery management system (BMS). Test battery performance under load to identify potential failures early. \\ \hline
         Software bugs causing unexpected robot behaviour & 3 & 5 & 15 & Mitigate & Implement thorough software testing and continuous integration, including unit tests, hardware-in-the-loop (HIL) testing, and simulation. \\ \hline
         Mechanical misalignment or interference between components & 2 & 4 & 8 & Mitigate & Use precise manufacturing techniques and perform tolerance checks during assembly. Maintain a detailed CAD model to predict issues. \\ \hline
         Network connectivity issues between robots or with the main controller & 3 & 4 & 12 & Mitigate & Implement a robust communication protocol with redundancy and conduct extensive testing in environments with potential interference. \\ \hline
         Difficulty in achieving real-time response due to computational limitations & 4 & 4 & 16 & Mitigate & Optimise code for performance, offload computationally heavy tasks to specialised hardware, and monitor performance metrics. \\ \hline
         Safety hazards from high-speed movement of robots & 3 & 4 & 12 & Mitigate & Establish a safe testing environment, use speed limiters during testing, and enforce safety protocols. Use emergency stop functionality in the system. \\ \hline
         Inconsistent performance in different environmental conditions (e.g., different flooring types, lighting conditions) & 2 & 5 & 10 & Mitigate & Conduct testing in various environments to fine-tune sensor configurations and control parameters. \\ \hline
         Software-hardware integration issues due to different update cycles & 3 & 4 & 12 & Mitigate & Maintain version control, ensure all subsystems are compatible, and establish regular integration meetings for synchronisation. \\ \hline
         Changes in competition rules or specifications during development & 2 & 5 & 10 & Mitigate & Monitor rule updates, keep designs flexible, and allocate resources for rapid changes when required. \\ \hline
    \end{tabularx}
    \caption{The risks, their probability, impact, risk score and the planned responses.}
    \label{tab:risk_response}
\end{table}

%-------------------------------------------------------------------------------

\begin{comment}
\helper{Example section\\}
We conduct a rigorous risk analysis and response planning as part of our project management. This involves identifying potential risks and assigning them a probability (from 1, very unlikely, to 5, very likely) and an impact score (from 1, minimal, to 5, critical). We then calculate a risk score by multiplying the probability by the impact. 

The risk score thresholds are interpreted as follows:
\begin{itemize}
\item 1-5: Low priority risks. These are monitored but may not require immediate action.
\item 6-10: Medium priority risks. These require a mitigation plan and should be addressed as resources allow.
\item 11-15: High priority risks. These require a detailed mitigation plan, and resources should be allocated to address these risks immediately.
\item 16-25: Critical risks. These must be addressed immediately with a detailed and efficient mitigation plan to avoid severe project disruption.
\end{itemize}

The table below outlines the identified risks, their evaluation, and planned responses:

\begin{table}[H]
    \centering
    \resizebox{\textwidth}{!}{%
    \begin{tabular}{|l|l|l|l|l|}
    \hline
    Risk &
      Probability(1 to 5) &
      Impact(1 to 5) &
      Risk(P*I) &
      Risk Response \\ \hline
    Critical team member leaves during project &
      2 &
      5 &
      10 &
      Develop contingency plan, cross-train team members \\ \hline
    Scope creep leading to project delay &
      4 &
      4 &
      16 &
      Regular scope review, robust change management process \\ \hline
    Technology becomes obsolete &
      1 &
      4 &
      4 &
      Regularly update technological skillsets, maintain flexibility in tech stack \\ \hline
    Software bugs detected after deployment & 3 & 5 & 15 & Rigorous testing process, allocate resources for post-deployment bug fixes \\ \hline
    Unexpected budget cuts &
      2 &
      5 &
      10 &
      Prepare a flexible budget that can accommodate cuts, regular financial reviews \\ \hline
    \end{tabular}%
    }
    \end{table}

The mitigation strategies and responses to these risks form an integral part of our project plan, ensuring that we are prepared for uncertainties and can effectively manage them should they arise.
\end{comment}

%===============================================================================