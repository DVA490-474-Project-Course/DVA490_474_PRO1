\newpage
%===============================================================================
\section{Risk analysis and response planning}
\label{section:risk_analysis_and_response_planning}

A thorough risk analysis was done and a response plan was created to ensure that any type of damage to the project is planned for and can be dealt with appropriately.
A table was created with the following attributes:
\begin{itemize}
    \item Potential risk: short description of a potential risk. (Only those which were deemed to have a notable impact will be covered).
    \item Probability: score from $1\text{-}5$, with $1$ being very unlikely, and $5$ being very likely.
    \item Impact: score from $1\text{-}5$, with $1$ having a very small impact, and $5$ having a very large impact.
    \item Risk score: score from $1\text{-}25$, with $1$ being a low priority risk, and $25$ being a critical priority risk.
    \item Strategy: which strategy is to be used to deal with the risk. There are four possible: accept, transfer, ignore, mitigate.
    \item Mitigation action: what the mitigation action entails.
\end{itemize}
The table can be seen in Table.\:\ref{tab:risk_response}.

\begin{table}
    \centering
    \resizebox{\columnwidth}{!}{\noindent\begin{tabular}{|c|c|c|c|c|c|} \hline
         \textbf{Potential risk}                            & \textbf{Probability}  & \textbf{Impact}   & \textbf{Risk score}   & \textbf{Strategy}     & \textbf{Mitigation action} \\ \hline
         Simulation differing substantially from reality    & 3                     & 3                 & 9                     & Accept                & N/A \\ \hline
    \end{tabular}}
    \caption{The risks, their probability, impact, risk score and the planned responses.}
    \label{tab:risk_response}
\end{table}

%-------------------------------------------------------------------------------

\begin{comment}
\helper{Example section\\}
We conduct a rigorous risk analysis and response planning as part of our project management. This involves identifying potential risks and assigning them a probability (from 1, very unlikely, to 5, very likely) and an impact score (from 1, minimal, to 5, critical). We then calculate a risk score by multiplying the probability by the impact. 

The risk score thresholds are interpreted as follows:
\begin{itemize}
\item 1-5: Low priority risks. These are monitored but may not require immediate action.
\item 6-10: Medium priority risks. These require a mitigation plan and should be addressed as resources allow.
\item 11-15: High priority risks. These require a detailed mitigation plan, and resources should be allocated to address these risks immediately.
\item 16-25: Critical risks. These must be addressed immediately with a detailed and efficient mitigation plan to avoid severe project disruption.
\end{itemize}

The table below outlines the identified risks, their evaluation, and planned responses:

\begin{table}[H]
    \centering
    \resizebox{\textwidth}{!}{%
    \begin{tabular}{|l|l|l|l|l|}
    \hline
    Risk &
      Probability(1 to 5) &
      Impact(1 to 5) &
      Risk(P*I) &
      Risk Response \\ \hline
    Critical team member leaves during project &
      2 &
      5 &
      10 &
      Develop contingency plan, cross-train team members \\ \hline
    Scope creep leading to project delay &
      4 &
      4 &
      16 &
      Regular scope review, robust change management process \\ \hline
    Technology becomes obsolete &
      1 &
      4 &
      4 &
      Regularly update technological skillsets, maintain flexibility in tech stack \\ \hline
    Software bugs detected after deployment & 3 & 5 & 15 & Rigorous testing process, allocate resources for post-deployment bug fixes \\ \hline
    Unexpected budget cuts &
      2 &
      5 &
      10 &
      Prepare a flexible budget that can accommodate cuts, regular financial reviews \\ \hline
    \end{tabular}%
    }
    \end{table}

The mitigation strategies and responses to these risks form an integral part of our project plan, ensuring that we are prepared for uncertainties and can effectively manage them should they arise.
\end{comment}

%===============================================================================