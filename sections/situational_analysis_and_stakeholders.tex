%===============================================================================
\section{Situational analysis and stakeholders}
\label{section:situational_analysis_and_stakeholders}

The situational analysis of this RoboCup project involves evaluating both internal and external factors that can impact the project development process. Understanding these important factors gives the team the ability to make accurate and strategical decisions to ensure a successful outcome.  


%===============================================================================
\subsection{SWOT-analysis}
The main reason for this analysis is to identify strengths, weaknesses, opportunities, and threats that stand behind the projects outcome.  This analysis helps the team identify areas of potential growth, manage risks, and capitalise on resources available through partnerships and institutional support. 


\textbf{Strengths:}
\begin{itemize}
    \item Skilled team: the project has well defined team structure with experience in various areas in both software and hardware. 
    \item Strong team collaboration: the team members strive to maintain great relationship with each other.   
    \item Advanced tools and technologies: use of advanced tools and technologies including \ac{ros2}, GoogleTest and tools such as OnShape for 3D-\acs{cad} modelling. 
\end{itemize}

\textbf{Weaknesses:}
\begin{itemize}
    \item Time constraints: the project operates within a fixed timeline, which limits the projects scope and flexibility. 
    \item No time for physical testing: due to the project having a fixed timeline, team members have limited opportunities to test the system in real-world conditions.
    \item Project budget: the project budget is limited and inhibits the teams options. 
    \item Large group experience: the team members lack experience working on a project with a group size this large.
\end{itemize}

\textbf{Opportunities:}
\begin{itemize}
    \item International collaboration: opportunity to add experienced members to the team with \ac{mdu}s international collaboration with \ac{udea} and \ac{utp}, possibly obtaining even greater results. 
    \item Learning and practice: significant potential for students to learn and use advanced technologies. 
\end{itemize}

\textbf{Threats:}
\begin{itemize}
    \item Technical issues: the simulation could differ from reality, which could result in unforeseen implications when deploying the physical system. 
    \item Demands imposed by the stakeholders and sponsors: stakeholders can and may limit the choices available to team members. This could include things like which tools and features are to be used or not used, all of which will affect the quality of project deliverables. 
\end{itemize}

\textbf{Conclusions and actions}
\begin{itemize}
   \item  The lack of experience working in large groups can be solved by dividing the team into smaller groups, each focusing on specific areas. This approach is beneficial because the project involves multiple working areas, making this division both practical and effective.
   \item The time constraint can be overcome through effective time management for all team members. A well-coordinated and skilled team can minimize delays by assigning tasks to the members with the most relevant experience, ensuring the right people handle the right tasks.
   \item International collaboration provides valuable opportunities for early design testing, which can help identify and resolve potential issues or delays before they become critical. By working with partners from different regions, teams can share diverse insights and expertise, ensuring that the design process is more thorough and efficient. 
\end{itemize}

%-------------------------------------------------------------------------------

\begin{comment}
\textbf{Conclusions:}
\begin{itemize}
    \item Need to secure additional funding and focus on continuous skills development
    \item Strategic alliances should be considered to mitigate weaknesses and threats
    \item Constant monitoring and lobbying required to navigate regulatory landscape
\end{itemize}
\end{comment}

%===============================================================================

\subsection{Stakeholder mapping}

Mapping and analysis of individuals, groups, and organisations that might affect or will be affected by the project.

\begin{itemize}
    \item \textbf{Internal Stakeholders:} Mikael Ekström and Emil Persson.
    \item \textbf{Investors:} \ac{mdu} has provided financial support for the project and invested in its success. 
    \item \textbf{Partners:} \ac{udea} and \ac{utp} working with \ac{mdu} on the project.
    \item \textbf{Customers:} Mikael Ekström, the project stakeholder and client, will utilise the autonomous robot once it is completed. 
    \item \textbf{Regulatory Bodies:} RoboCup and Swedish laws set the standards and rules which the project must comply with. 
    \item \textbf{Suppliers:} \ac{mdu} supplies the necessary materials and resources for the project. 
    \item \textbf{Competitors:} Other universities working on RoboCup projects, developing related technologies and solutions. 
\end{itemize}

%===============================================================================