%===============================================================================
\section{Background}
\label{section:background}

The \ac{ssl}-RoboCup is a tournament where autonomous robots play football\:\cite{da_silva_costa_multi-robot_2024}. Robocup aims to advance the state of the art for intelligent robots\:\cite{noauthor_small_nodate}. There are two different divisions, A and B, with A being the division for more advanced teams\:\cite{noauthor_rules_2024}. Divison A plays on a $12\:\text{m} \times 9\:\text{m}$ field with eleven robots\:\cite{da_silva_costa_multi-robot_2024},\cite{noauthor_rules_2024}. Meanwhile, division B plays with only six robots on a $9\:\text{m} \times 6\:\text{m}$ field\:\cite{da_silva_costa_multi-robot_2024},\cite{noauthor_rules_2024}. 
Each robot needs to be designed to fit inside a cylinder with a diameter of $0.18\:\text{m}$ and a height of $0.15\:\text{m}$\:\cite{noauthor_rules_2024}. The robots are allowed to have both a kicking and a dribbling device\:\cite{noauthor_rules_2024}. The ball used is an orange golf ball which is not allowed to be kicked so that it moves faster than $6.5\:\text{m}/\text{s}$ \cite{noauthor_rules_2024}. 
An \ac{ssl}-RoboCup match consists of first half, half-time break and second half, each one lasting for a period of $300\:\text{s}$\:\cite{noauthor_rules_2024}.

%-------------------------------------------------------------------------------

%\helper{Description with a clear and defined connection to the goals. It is advisable to connect to any related project in the background description.}

%===============================================================================