\newpage
%===============================================================================
\section{Communication plan}
\label{section:communication_plan}

The communication plan was established to guarantee that the correct target audience receives the necessary information at the right time and through appropriate channels, see Table.\:\ref{tab:communication_plan}.

\begin{table}
    \centering
    \resizebox{\columnwidth}{!}{\noindent\begin{tabular}{|c|c|c|c|c|c|} \hline
         \textbf{Target audience}   & \textbf{Purpose}      & \textbf{Type of information}  & \textbf{Timing}   & \textbf{Communication channels}   & \textbf{Responsible}  \\ \hline
         Project members            & Everything            & Everything                    & Any time          & Discord                           & Project members       \\ \hline
         Project members            & Keep the team updated & Project status \& information & Weekly            & Meeting                           & Team Leader           \\ \hline
         Stakeholders & Inform stakeholders of status of project, supervision \& aid & Project progress summary, project plan, on leave, feedback and supervising& Weekly & Meeting & Team Leader \\ \hline
         Colombia contact person & Collaboration & Task assignment, progress report & Weekly & Microsoft teams and mail & Team Leader \\  \hline
         Panama contact person & Collaboration & Task assignment, progress report & Weekly & Microsoft teams and mail & Team Leader \\  \hline
    \end{tabular}}
    \caption{The communication plan for the project.}
    \label{tab:communication_plan}
\end{table}

%------------------------------------------------------------------------------

\begin{comment}
\helper{Plan for spreading information in the purpose of guaranteeing the right target group gets the right information at the right time and through the right channels, simple example listed below.}

The Communication Plan is essential to any project, serving as a guide for sharing information throughout the project lifecycle. It optimizes the distribution of project-related data among team members and stakeholders, fostering cooperation, ensuring transparency, and promoting mutual understanding.

This strategic document lays out various facets of communication, such as the type of information that needs to be disseminated, the audience for each type of information, the timing or frequency of communication, and the mediums or channels for transmitting the information.

\begin{itemize}
\item \textbf{Who (Target Audience):} This determines who needs to receive specific information. This could include project team members such as the hardware and software developers, hardware and software team leaders, stakeholders, the examiner, the course responsible, and the supervisor.
\item \textbf{Why (Purpose):} This refers to the reason behind the communication. For example, project updates inform stakeholders about progress, and task assignments may be needed to guide the developers. At the same time, risk alerts might be necessary to manage potential issues.
\item \textbf{What (Type of Information):} This refers to the specific data or information to be shared. This could include project updates, task assignments, meeting agendas, change requests, and risk alerts.
\item \textbf{When (Timing):} This indicates when and how often the communication should occur. This could be daily, weekly, biweekly, monthly, or at specific project milestones, depending on the nature of the information.
\item \textbf{How (Communication Channels):} This determines how the information will be delivered. Communication methods include email, direct communication, meetings, project management software, or the course management system.
\item \textbf{Responsible:} This identifies who is responsible for ensuring the communication happens. This can vary depending on the type of information and the target audience.
\end{itemize}

By implementing a well-structured communication plan, potential misunderstandings can be minimized, efficient decision-making can be facilitated, and a conducive environment for project success can be established. This is achieved through setting clear communication protocols and procedures that are known and understood by everyone involved in the project.

The following example shows a simple communication plan for a project. This table is just a template and can be modified to fit your project’s needs, try to be more specific and who it may concern:

\begin{table}[H]
    \centering
    \resizebox{\textwidth}{!}{%
    \begin{tabular}{|l|l|l|l|l|l|}
    \hline
    \textbf{Who}   & Why                               & What             & When                & How                                & Responsible              \\ \hline
    HW Team Leader & To keep updated on project status & Project Updates  & Weekly              & Email, Project Management Software & Project Team             \\ \hline
    SW Team Leader & To keep updated on project status & Project Updates  & Weekly              & Email, Project Management Software & Project Team             \\ \hline
    HW Developer   & To know their assignments         & Task Assignments & As needed           & Project Management Software        & HW and SW Team Leaders   \\ \hline
    SW Developer   & To know their assignments         & Task Assignments & As needed           & Project Management Software        & HW and SW Team Leaders   \\ \hline
    HW Team Leader & To prepare for meetings           & Meeting Agendas  & Before each meeting & Email                              & Meeting Organiser        \\ \hline
    SW Team Leader & To prepare for meetings           & Meeting Agendas  & Before each meeting & Email                              & Meeting Organiser        \\ \hline
    HW Developer   & To adapt to changes               & Change Requests  & As needed           & Meetings, Email                    & Person Requesting Change \\ \hline
    SW Developer   & To adapt to changes               & Change Requests  & As needed           & Meetings, Email                    & Person Requesting Change \\ \hline
    HW Team Leader & To manage risks                   & Risk Alerts      & As needed           & Direct Communication, Email        & Risk Manager             \\ \hline
    SW Team Leader & To manage risks                   & Risk Alerts      & As needed           & Direct Communication, Email        & Risk Manager             \\ \hline
    \end{tabular}%
    }
\end{table}
\end{comment}

%===============================================================================