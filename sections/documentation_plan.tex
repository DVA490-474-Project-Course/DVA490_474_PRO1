\newpage
%===============================================================================
\section{Documentation plan}
\label{section:documentation_plan}

Good documentation is a critical tool to help project management since it allows for reviewing the work done and observing the progress. However it also serves other very important purposes like transparency, accountability and helps others understand, use and contribute to the project. To this end, the following documentation plan was established for this project.

%------------------------------------------------------------------------------

\begin{comment}
\helper{Plan for how the project’s documentation is to be managed. Example listed below.}

The purpose of a Documentation Plan is to provide a structured framework for creating, storing, reviewing, and disseminating project-related information. It ensures systematic recording and retention of all details, facilitating efficient communication, transparency, accountability, and continuity throughout the project. Comprehensive documentation is a critical tool in project management, aiding in tracking progress, making informed decisions, and providing references for future initiatives. Furthermore, it promotes knowledge sharing and learning during the project and beyond.

The following sections present a detailed example of a Documentation Plan. It outlines what needs to be documented, how and when to do so, who is responsible for various documentation tasks, where the documents will be stored, and the review and approval process for these documents. This example serves as a guide and should be tailored to fit your project's specific needs and circumstances.
\end{comment}

%===============================================================================
\subsection{What to Document}

The following items will be documented:
\begin{itemize}
    \item Requirements: available in this document, see Requirement section\:\ref{section:requirement}.
    \item Specifications: available in this document, see Tools, Technologies, and Systems section\:\ref{subsection:tools_technologies_and_systems}.
    \item Design: available on the projects GitHub.
    \item Changes:
    \begin{itemize}
        \item Software: available from Git commit messages and in changelog in each repositories docs directory on the projects GitHub.
        \item Hardware: each commit to the repository contains information about changes made.
    \end{itemize}
    \item Issues/Known bugs:
    \begin{itemize}
        \item Software: available from Git commit messages and in changelog in each repositories docs directory on the projects GitHub.
        \item Hardware: issues can be found in the issues tab for the repository.
    \end{itemize}
    \item Testing: 
    \begin{itemize}
        \item Software: record of all test and simulation results available in each repositories wiki on the projects GitHub.
        \item Hardware: data from standardised tests will be documented in the wiki.
    \end{itemize}
    \item User manual:
    \begin{itemize}
        \item Software: available in user\_guide in each repositories docs directory on the projects GitHub.
        \item Hardware: the repositories wiki contains the user manual.
    \end{itemize}
\end{itemize}

The software team will have the following documentation:
\begin{itemize}
    \item Changelog: record major changes, new features, bugs, bug fixes and removed features in each release version.
    \item \acs{api}: document the functions, classes and modules of the project.
    \item User Guide: installation instructions, how to run and use the application, configuration options.
    \item README: a multitude of different README files will exist with different documentation purposes.
    \item Code comments: improve understanding of the source code.
    \item Git commit comments: log changes and track the status of the code and project.
    \item GitHub wiki: 
    \begin{itemize}
        \item test results: create a record of the test results.
        \item simulation results: create a record of simulation results.
    \end{itemize}
\end{itemize}

The hardware team will have the following documentation:
\begin{itemize}
    \item Changelog: design changes, wiki updates, new features, bugs, bug fixes.
    \item Setup: A guide for installing the required dependencies, required hardware and how to configure it.
    \item Component database: The wiki contains information about each component mentioned in the \ac{bom}.
\end{itemize}

%------------------------------------------------------------------------------

%Define the types of documentation that will be required. This could include design documents, technical specifications, meeting minutes, code comments, testing reports, user manuals, troubleshooting guides, etc.

%===============================================================================
\subsection{How to Document}

All documentation is to be in British English and must follow scientific writing standards.

%------------------------------------------------------------------------------

%Specify the format and structure of each document. For example, technical specifications might require diagrams and detailed descriptions, while meeting minutes may be more informal. Develop templates for consistency.

%===============================================================================
\subsection{When to Document}

It is paramount that documentation takes place the moment any work has been done or any changes have been made. For complete clarity:
\begin{itemize}
    \item Requirements: documentation of any changes or additions to the requirements should be done instantly.
    \item Specifications: if any specifications are subject to change then the accompanying documentation must instantly be altered.
    \item Design: whenever there is an alteration of the design, the corresponding design files and documentation must be updated accordingly as soon as possible.
    \item Changes: all changes must be logged.
    \item Issues/Known bugs: when there is an update to an existing issue or bug, or a new one is found, then these must instantly be documented.
    \item Testing: any changes to existing tests, addition of new tests and test results must instantly be documented.
    \item User manual: the user manual must only be documented in time for the final release and whenever any post release updates are done.
\end{itemize}

%------------------------------------------------------------------------------

%Define when each document should be updated. For instance, design documents should be updated whenever a design decision is made, testing reports after every test run, and meeting minutes following each meeting.

%===============================================================================
\subsection{Who is Responsible}

Team leads are responsible for overseeing their respective teams documentation as well as ensuring all the proper documentation is available and in proper order. Developers are responsible for producing proper documentation for everything they have done.

%------------------------------------------------------------------------------

\begin{comment}
Assign documentation duties to specific roles:

\begin{itemize}
\item \textbf{Hardware Team Leader:} Oversees and approves hardware-related documentation and assigns hardware-related documentation tasks.
\item \textbf{Software Team Leader:} Oversees and approves, and assigns software-related documentation tasks.
\item \textbf{Hardware Developer:} Creates and updates hardware-related documentation as assigned.
\item \textbf{Software Developer:} Creates and updates software-related documentation as assigned.
\end{itemize}
\end{comment}

%===============================================================================
\subsection{Where to Store Documents}

All software and hardware documentation will be stored on the projects \href{https://github.com/DVA490-474-Project-Course}{GitHub}\footnote{https://github.com/DVA490-474-Project-Course} page.

%------------------------------------------------------------------------------

%Define where documents will be stored. This could be on a shared drive, a project management tool, or a version control system. Ensure it is accessible to all team members and has appropriate backup and security measures.

%===============================================================================
\subsection{Document Review Process}

The software and hardware lead will continuously review their respective teams documentation.

%------------------------------------------------------------------------------

%Specify a process for reviewing and approving documents. This could involve peer reviews, leader reviews, or formal approval processes.

%===============================================================================
\subsection{Training}

Members were introduced to the documentation plan during the initial project period.

%------------------------------------------------------------------------------

%Ensure everyone involved in the project is aware of the documentation plan and trained on how to use the tools and templates. This can be part of the initial project orientation or documentation training session.

%===============================================================================