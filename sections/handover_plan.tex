%===============================================================================
\section{Handover plan}
\label{section:handover_plan}

The handover plan outlines all necessary steps for successfully transferring the project deliverables to the stakeholders. It will include all project documentation, code and assets, ensuring the ability for maintenance and further development. 

%------------------------------------------------------------------------------

\begin{comment}
\helper{How to deliver the product to the client and implement it into the environment it is meant for. The following subsections is an general simple example}

\noindent This Handover Plan outlines the systematic process of transferring all project deliverables, documentation, tools, and knowledge to the project owner at the end of the project or a project phase. The objective is to ensure a smooth transition, preserving all the hard work done on the project and providing a firm basis for future work. Understanding this plan will ensure that all team members know what is required in the final stages of the project.
\end{comment}

%===============================================================================
\subsection{Handover Methodology}

The projects handover methodology includes the following:
\begin{itemize}
    \item Presentation: prior to the handover, a presentation will be held. This will help familiarise the client with the project and the results obtained. This presentation should ease the integration of new staff to the project but the documentation alone should be a sufficient resource for them to learn how to use the product after the handover process is complete.
    \item Final testing: a complete testing session where every single test will be run and all the results are logged. This is critical for a transparent description of the state of the product at the time of the handover.
    \item Final documentation review: a review of all the documentation, ensuring everything is available, accessible and in order. The documentation is a vital source of information, enabling for future development and helping users use the product.
    \item Hardware inventory check: a final comprehensive hardware inventory check will take place. Ensuring an accurate description of the final product and the process of creating it.
    \item Software check: a final check of the correctness and availability of all the developed code and software material on the projects GitHub. Ensuring further development can proceed smoothly and allowing for easy on boarding of new contributors with all the material they might need gathered on the GitHub.
\end{itemize}

\begin{comment}
The handover will be structured, clear and simple to follow ensuring easy accessibility to the stakeholders. \\
The handover methodology is will take form of: \\
{* Canvas:} Handing in the final project report.\\  
{* GitHub:} Handing in all code, documentation, issues, bugs and all other project files.\\ 
{* Overleaf:} Will be used for the final project report. \\ 

Handover will also include all necessary documentation, such as user manuals, system design, API references and test results. 
\end{comment}

%------------------------------------------------------------------------------

%Our handover methodology includes [describe methods and steps such as inventory check, final documentation, presentation, and sign-off]. These steps have been chosen to ensure a comprehensive and efficient handover process.

%===============================================================================
\subsection{Handover Team Structure and Roles}
\label{subsection:handover_team_structure_and_roles}

The leading roles will be in charge of the handover process and will have the following responsibilities:
\begin{itemize}
    \item \textbf{Team Leader:} Responsible for overseeing the entire handover process, ensuring that all necessary materials are included and that the product is delivered before the deadline.
    \item \textbf{Software Lead:} Responsible for GitHub and all software related material, including documentation and proper delivery of all the material. 
    \item \textbf{Hardware Lead:} Responsible  for all hardware assets, designs and documentation, and their delivery.
\end{itemize}

%------------------------------------------------------------------------------

\begin{comment}
Our handover team includes the following roles:

\begin{itemize}
\item \textbf{Handover Team Leader:} Coordinates the handover process, ensuring that all project deliverables and documentation are transferred wholly and accurately.
\item \textbf{Handover Specialist:} Conducts presentations (if required), answers queries about the project, and ensures that the project owner is comfortable with understanding the project's results and operation.
\end{itemize}
\end{comment}

%===============================================================================
\subsection{Handover Tools, Technologies, and Systems}

The tools which will be used during the handover process includes:
\begin{itemize}
    \item Canvas: Canvas will be used to deliver the project report.
    \item GitHub: GitHub will be used to deliver all code, documentation and all other project files.
    \item Overleaf: Overleaf will be used to conduct presentations and the writing of all papers.
\end{itemize}

%------------------------------------------------------------------------------

\begin{comment}
We leverage the following tools, technologies, and systems during our handover process:

\begin{itemize}
\item \textbf{Inventory Management Tools:} We utilize [Tool/Platform Name, version] to catalogue and track all project deliverables for handover, which can be downloaded from [source].
\item \textbf{Presentation Tools:} If necessary, we use [Tool/Platform Name, version] to conduct presentations or demonstrations, and it can be downloaded from [source].
\end{itemize}

Please install the correct version numbers as specified for consistency and compatibility.
\end{comment}

%===============================================================================
\subsection{Handover Schedule}

The handover schedule consists of an initial handover on January $10\text{th}$ $2025$ and a followup handover on February $14\text{th}$ $2025$ if the product from the initial handover was found unsatisfactory.

%------------------------------------------------------------------------------

%Our handover schedule aligns with the overall project timeline. The detailed handover schedule can be found in our project timeline [footnote or ref to document].

%===============================================================================
\subsection{Documentation}

Comprehensive documentation regarding all the project information including user guides, design, changelog, \ac{api}, test and simulation results can all be accessed on the projects \href{https://github.com/DVA490-474-Project-Course}{GitHub}\footnote{https://github.com/DVA490-474-Project-Course} page.

%------------------------------------------------------------------------------

%Comprehensive project documentation, including user manuals, technical documentation, and project reports, will be provided during the handover process. These documents can be accessed from [footnote or ref to Document Storage/Management System].

%===============================================================================
\subsection{Presentation and Demonstration}

Presentations and demonstrations will be provided prior to the final hand-over to inform the sponsor of the current state of the project, as well as the results obtained. The preliminary presentation date is December $12\text{th}$ $2024$.

%------------------------------------------------------------------------------

%Our team will conduct presentations and demonstrations where necessary to familiarize the project owner with the project's operation and results.

%===============================================================================
\subsection{Final Sign-off}

A final sign-off will occur once the project sponsor has reviewed the delivery, and ensured that all project requirements and goals are met.
The final sign-off will act as a formal acceptance where the stakeholders confirm their satisfaction with the project outcomes. This will mark the official conclusion of the project, after which the team will be relieved of all responsibilities concerning the product and the project.

%------------------------------------------------------------------------------

%The project owner will sign off, indicating successful handover and completion. This will be done using [describe sign-off method/process].

%===============================================================================
\subsection{Version Control for Handover Artifacts}

This project will use the version control tool Git, see Version Control section\:\ref{subsection:version_control}.

%------------------------------------------------------------------------------

%We utilize [Version Control System Name, e.g., Git] to manage changes to our handover artefacts and maintain different versions. For detailed guidance on using this system, please refer to our [footnote or ref to version control usage guide].

%===============================================================================
\subsection{Follow-up and Feedback}

Any feedback is greatly appreciated and will be addressed during "Hand-in after deadline and revision" if the product is not found satisfactory. This will ensure that a satisfactory product is delivered and that all concerns are put to rest.

%------------------------------------------------------------------------------

%We highly value feedback from the project owner and have a process for follow-up [describe follow-up process] to ensure that all aspects of the handover were completed satisfactorily and address any potential concerns post-hand-over.

%===============================================================================